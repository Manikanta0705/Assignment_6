\documentclass[journal,12pt,twocolumn]{IEEEtran}

\usepackage{setspace}
\usepackage{amssymb}
\usepackage{amsthm}
\usepackage{mathrsfs}
\usepackage{longtable}
\usepackage{enumitem}
\usepackage{mathtools}
\usepackage{longtable}
\usepackage[breaklinks=true]{hyperref}

\usepackage{listings}
    \usepackage{color}                                            %%
    \usepackage{array}                                            %%
    \usepackage{longtable}                                        %%
    \usepackage{calc}                                             %%
    \usepackage{multirow}                                         %%
    \usepackage{hhline}                                           %%
    \usepackage{ifthen}                                           %%
    \usepackage{lscape}     
    \usepackage{amsmath}
       
\lstset{
%language=C,
frame=single, 
breaklines=true,
columns=fullflexible
}
\def\inputGnumericTable{}

\bibliographystyle{IEEEtran}
\providecommand{\pr}[1]{\ensuremath{\Pr\left(#1\right)}}
\providecommand{\brak}[1]{\ensuremath{\left(#1\right)}}

\newcommand{\question}{\noindent \textbf{Question: }}
\newcommand{\solution}{\noindent \textbf{Solution: }}

\title{Assignment 6}
\author{MANIKANTA UPPULAPU (BT21BTECH11005)}

\begin{document}
    % make the title area
    \maketitle  
    
    \question If P(A) = 1/2, P(B) = 0, then P (A/B) is
    \begin{enumerate}
        \item 0 
        \item 1/2
        \item not defined
        \item 1
    \end{enumerate}
    
    \solution 
     If $X$ and $Y$ are two events in a sample space $S$, then The Conditional Probability of $X$ given $Y$ is defined as
        \begin{align}
        \label{1}  \pr{X \mid Y} &= \frac{\pr{XY}}{\pr{Y}}  
        \end{align}
        Given, $A$ and $B$ are the events such that :
        %inserting the "Table" from "Tables" folder
        \begin{table}[ht!]
        \centering
           \input{Table}
        \caption{Given Data}
	        \label{Tables:Table}
        \end{table}
        \begin{enumerate}
             Using \eqref{1},
            \begin{align}
                &\pr{A \mid B} = \frac{\pr{AB}}{\pr{B}}\label{2} \\
                \implies &\pr{A \mid B} = \frac{{\pr{AB}}}{{0}}\label{3} \\
                \implies &\fbox{\pr{A \mid B} = \text{not defined}}\label{4} 
            \end{align}
            \end{enumerate}
            \end{document}
